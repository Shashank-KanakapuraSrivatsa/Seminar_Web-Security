\chapter{Defence against HTTP Desync Attacks}
The author in \cite{b6} has not only proposed techniques to carry out HTTP Desync Attacks but has also suggested the defence mechanisms or preventive measures to protect a website from such attacks. We mention a few of them here :
\begin{itemize}
	\item Lesser number of layers decrease the risk of desynchronization attacks. Websites not containing load balancers, CDNs and reverse proxies are almost immune to desynchronization. 
	\item Forcing the frontend servers to use HTTPS/2 \cite{b15} while communicating with backend. 
	\item Disabling backend connection reuse.
	\item Same webserver software with identical configuration can be used to run all the servers in a chain. 
	\item Frontend server can be configured to normalize ambiguous requests and then route them forward. 
	\item Rejecting ambiguous requests at the backend and dropping the corresponding connection is one option but is not viable as it affects the regular traffic. 
	\item There are tools which can correct the \textsc{Content-Length} header before sending requests. These can be effective against request smuggling. 
\end{itemize}
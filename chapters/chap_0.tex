\begin{abstract}
	Internet has become an integral part of the modern world and it is extremely difficult to imagine a world without internet. One of the most significant activity on the internet is exchange of data between computers, generally referred to as Clients and Servers. There are a huge number of formats in which the data can be transferred over internet and many new formats are being introduced. To organize and govern such data transfers, several protocols have been defined by organisations. One of the prominent protocol is Hyper Text Transfer Protocol(HTTP), where a Client submits a request to the Server over the internet and the Server responds to the Client with approriate data or message.\\
	Like several other protocols, HTTP also contains some vulnerabilities which can be exploited by malicious attackers and thus it becomes extremely necessary to identify such vulnerabilities. Security experts or \textit{White hat hackers} help to identify vulnerabilities present in protocols such as HTTP and notify the concerning organisations or individuals about the same. This helps to fix the vulnerabilities present and thus prevent information leakage. There are several vulnerabilities that have been identified in the past and many of these problems are already addressed. In this report, we talk about a relatively new technique known as 'HTTP Desync Attack' which is developed to exploit the vulnerability in HTTP protocol. In this technique, the attacker tricks the Server into believing that the malicious request is actually a part of a normal user`s request and thus allows the attacker to gain control over the request. This report reviews HTTP Desync Attacks in detail and further evaluates it to understand the pros and cons of the technique. 
	
\end{abstract}


\cleardoublepage

\chapter{Background}
\section{HTTP} 
Hyper Text Transfer Protocol(HTTP) \cite{b1} is a stateless application layer protocol meant for distributed, collaborative, hypertext information systems. The exchange of data between a client and a server is governed by HTTP. A client sends a HTTP request which can contain standard as well as user's custom data together. The client receives this request and decodes the same. Based on the request the server decides to either perform an action or send data back to the client which can contain requested information or server messages.  

\section{HTTP 1.1}
HTTP 1.1 \cite{b1} is an improvised version of HTTP which succeeds the version HTTP 1.0. This version includes several improvements, few of which are :
\begin{itemize}
	\item support for keep-alive feature where a connection can be re-used.
	\item pipelining
	\item chunked responses
\end{itemize} 

\section{HTTP Request}
A HTTP Request \cite{b7} is a message sent by the client, to the server, to invoke an action at the server side.

\section{HTTP Response}
A HTTP Response \cite{b7} is a message sent back by the server, to the client, which contains the requested information, resource and status code. 

\section{HTTP Request Methods}
HTTP Request Methods \cite{b8} are verbs which specify the action that has to be performed by the server. The popular methods are :
\begin{itemize}
	\item GET : To request data from the resource.
	\item POST : To submit information to a resource. 
	\item PUT : To submit information to a resource but replaces the existing information. 
	\item DELETE : Deletes the specified resource. 
\end{itemize}

\section{HTTP Headers}
Headers contain meta-deta which corressponds to the request, request method and the information being communicated. There are a vast number of HTTP Headers and they are documented by Mozilla in \cite{b17}.\\
\textsc{Transfer-Encoding: chunked} and \textsc{Content-Length} are the main HTTP Headers used in this report. \\
\begin{itemize}
	\item \textbf{Transfer-Encoding: chunked} - Specifies that the chunked form of encoding is used to safely transfer the data.
	\item \textbf{Content-Length}
\end{itemize}
\section{HTTP Body}
It is the information being transmitted and the response message received.
\chapter{Conclusion}
Desynchronization attacks can be very effective in a chained system environment and can cause substantial damage. This fact has been established throughout this report. Even though this technique was first discussed way back in 2005 by WatchFire \cite{b5}, it was ignored for a very long time owing to its complexity. However, due to the efforts of the author in \cite{b6} this technique has again gained popularity and it has now been proved to be practically realizable. In this report, I have initially tried to provide some background information about HTTP and its headers. The entire agenda of this report is to provide an insight to HTTP Desynchronization attacks. In various sections of this report, we have understood the basic core concepts of request smuggling. A step-wise approach to carry out a desynchronization attack has been explained elaborately with atleast one example or case study. Several attack methods to reap the benefits of desynchronization have also been highlighted. Such information can be very effective to identify whether a given system is vulnerable to desynchronization attacks. \\
Subsequently, it is quite crucial to protect our systems from request smuggling. In this direction, several suggestions have been put forward by the author in \cite{b6} which can be suitably implemented to prevent such attacks. This report intends to enlighten the readers about HTTP Desynchronization attacks and help prevent any mischievious actions by attackers and also further secure websites. 